\documentclass[12pt]{article}
\usepackage[margin=1in]{geometry}
\usepackage[all]{xy}
\usepackage{amsmath,amsthm,amssymb,color,latexsym}
\usepackage{geometry}
\usepackage{physics}
\usepackage{mathtools}
\geometry{letterpaper}    
\usepackage{graphicx}
\usepackage{nicefrac}
\usepackage{enumitem}
\usepackage{minted}
\usepackage{tabularx,ragged2e}
\newcolumntype{x}{>{\small\raggedright\arraybackslash}X}
\usepackage{cancel}
\usepackage{multirow}
\usepackage{hyperref}
\hypersetup{
    colorlinks=true,
    urlcolor=cyan,
    }

    

\begin{document}
\noindent VASP to QE tables \hfill Andres Covarrubias    \\
5/23/2022

\hrulefill
%%%%%%%%%%%%%%%%%%%%%%%%%%%%%%%%% VASP OBJECTS %%%%%%%%%%%%%%%%%%%%%%%%%%%%%%%%%%%
\section*{VASP Objects}
All VASP objects are sourced from the \href{https://pymatgen.org/pymatgen.io.vasp.html?highlight=io\%20vasp\#module-pymatgen.io.vasp}{pymatgen.io.vasp} package of pymatgen except for Vasprun, which is a stand alone python project
\begin{center}
\begin{table}[ht]
\begin{tabularx}{\linewidth}{|>{\RaggedRight}p{2.5cm}|x|x|}\hline
 VASP version & Notes & QE Replacement \\ \hline
 
 %%% MPRelaxSet %%%
 \href{https://pymatgen.org/pymatgen.io.vasp.sets.html?highlight=mprelaxset#pymatgen.io.vasp.sets.MPRelaxSet}{MPRelaxSet} & 
 % Notes
 Implementation of VaspInputSet utilizing parameters in the public Materials Project. Typically, the pseudopotentials chosen contain more electrons than the MIT parameters, and the k-point grid is ~50\% more dense. The LDAUU parameters are also different due to the different psps used, which result in different fitted values. & 
 % QE Replacement
\href{https://pymatgen.org/pymatgen.io.pwscf.html?highlight=pwscf}{pwscf} module with scf kwargs\\ \hline
 
 %%% MPStaticSet %%%
 \href{https://pymatgen.org/pymatgen.io.vasp.sets.html?highlight=mpstaticset#pymatgen.io.vasp.sets.MPStaticSet}{MPStaticSet} &
  % Notes
Creates input files for a static calculation. &
 % QE Replacement
\href{https://pymatgen.org/pymatgen.io.pwscf.html?highlight=pwscf}{pwscf} module with nscf kwargs \\ \hline
 
 %%% MPSOCSet %%%
 \href{https://pymatgen.org/pymatgen.io.vasp.sets.html?highlight=mpsocset#pymatgen.io.vasp.sets.MPSOCSet}{MPSOCSet} &
  % Notes
 An input set for running spin-orbit coupling (SOC) calculations. &
 % QE Replacement
\href{https://pymatgen.org/pymatgen.io.pwscf.html?highlight=pwscf}{pwscf} module with SOC kwargs \\ \hline
 
 %%% Vasprun %%%
 \href{https://vasprun-xml.readthedocs.io/en/latest/}{Vasprun} &
  % Notes
 A python project used for quick analysis of VASP calculation solely from vasprun.xml &
 % QE Replacement
 \href{https://github.com/maxhutch/qe-tools/blob/master/README.md}{qe-tools} or command line scripting\\ \hline
\end{tabularx}
\end{table}
\end{center}

\newpage
%%%%%%%%%%%%%%%%%%%%%%%%%% VASP BUILT INS %%%%%%%%%%%%%%%%%%%%%%%%%%%%%%%%%%%%%%
\section*{VASP Built-ins}
\begin{center}
\begin{table}[ht]
\begin{tabularx}{\linewidth}{|>{\RaggedRight}p{2.5cm}|x|x|}\hline
 VASP  & Notes & QE \\ \hline

  %%% Chgcar %%%
 \href{https://www.vasp.at/wiki/index.php/CHGCAR}{Chgcar} &
  % Notes
 The CHGCAR file stores the charge density and the PAW one-center occupancies and can be used for restarting VASP calculations &
 % QE Replacement
Parse QE output using \href{https://github.com/nomad-coe/nomad-parser-quantum-espresso}{NOMAD}, Pymatgen, or GREP  \\ \hline
 
 %%% Oszicar %%%
 \href{https://www.vasp.at/wiki/index.php/OSZICAR}{Oszicar} &
  % Notes
 Information about convergence speed and about the current step is written to stdout and to the OSZICAR file. &
 % QE Replacement
Parse QE output using \href{https://github.com/nomad-coe/nomad-parser-quantum-espresso}{NOMAD}, Pymatgen, or GREP   \\ \hline
 
 %%% Outcar %%%
 \href{https://www.vasp.at/wiki/index.php/OUTCAR}{Outcar} &
  % Notes
 The OUTCAR file gives detailed output of a VASP run &
 % QE Replacement
Parse QE output using \href{https://github.com/nomad-coe/nomad-parser-quantum-espresso}{NOMAD}, Pymatgen, or GREP  \\ \hline
 
 %%% Potcar %%%
 \href{https://www.vasp.at/wiki/index.php/POTCAR}{Potcar} &
  % Notes
 The POTCAR file essentially contains the pseudopotential for each atomic species used in the calculation. If the number of species is larger than one, one simply concatenates the POTCAR files of the species. &
 % QE Replacement
\href{https://www.quantum-espresso.org/Doc/INPUT_PW.html#idm1484}{pseudo\_dir} in the \&control card, and specifying the pseudopotential file for each element specifically in the \&ATOMIC\_SPECIES card\\ \hline
\end{tabularx}
\end{table}
\end{center}
\newpage

%%%%%%%%%%%%%%%%%%%%%%%%%%% INPUT FILE PARAMETERS %%%%%%%%%%%%%%%%%%%%%%%%%%%%%%%%
\section*{Input File Parameters}
\begin{center}
\begin{table}[ht]
\begin{tabularx}{\linewidth}{|>{\RaggedRight}p{2.5cm}|x|x|}\hline
 VASP  & Notes & QE \\ \hline
 
 
 %%%%%%%%%%%%% KPAR %%%%%%%%%%%%%
 \href{https://www.vasp.at/wiki/index.php/KPAR}{KPAR} &
  % Notes
 Number of k-points that are to be treated in parallel &
 % QE Replacement
No direct replacement, see \href{https://www.quantum-espresso.org/Doc/user_guide/node17.html}{parallelism} section in QE user guide \\ \hline

 %%%%%%%%%%%%% NCORE %%%%%%%%%%%%%
 \href{https://www.vasp.at/wiki/index.php/NCORE}{NCORE} &
  % Notes
 Determines the number of compute cores that work on an individual orbital &
 % QE Replacement
 No direct replacement, see \href{https://www.quantum-espresso.org/Doc/user_guide/node17.html}{parallelism} section in QE user guide \\ \hline

 %%%%%%%%%%%%% NSIM %%%%%%%%%%%%%
 \href{https://www.vasp.at/wiki/index.php/NSIM}{NSIM} &
  % Notes
 Sets the number of bands that are optimized simultaneously by the RMM-DIIS algorithm. &
 % QE Replacement
 \href{https://www.quantum-espresso.org/Doc/INPUT_PW.html#idm1487}{diago\_rmm\_ndim} in \&electrons card, need to set  \\ \hline

 %%%%%%%%%%%%% SYMPREC %%%%%%%%%%%%%
 \href{https://www.vasp.at/wiki/index.php/SYMPREC}{SYMPREC} &
  % Notes
 Determines to which accuracy the positions of the atomic sites must be specified &
 % QE Replacement
 Specify accuracy in \&ATOMIC\_POSITIONS card \\ \hline
 
 %%%%%%%%%%%%% ISMEAR %%%%%%%%%%%%%
 \href{https://www.vasp.at/wiki/index.php/ISMEAR}{ISMEAR} &
  % Notes
 Determines how the partial occupancies $f_{nk}$ are set for each orbital. See documentation for different types of smearing options &
 % QE Replacement
 \href{https://www.quantum-espresso.org/Doc/INPUT_PW.html#idm382}{smearing} in \&system card \\ \hline
 
 %%%%%%%%%%%%% SIGMA %%%%%%%%%%%%%
 \href{https://www.vasp.at/wiki/index.php/SIGMA}{SIGMA} &
  % Notes
 Specifies the width of the smearing in eV. &
 % QE Replacement
 \href{https://www.quantum-espresso.org/Doc/INPUT_PW.html#idm406}{q2sigma} in \&system card \\ \hline
 
 %%%%%%%%%%%%% ISIF %%%%%%%%%%%%%
 \href{https://www.vasp.at/wiki/index.php/ISIF}{ISIF} &
  % Notes
 Determines whether the stress tensor is calculated and which principal degrees-of-freedom are allowed to change in relaxation and molecular dynamics runs. &
 % QE Replacement
 \href{https://www.quantum-espresso.org/Doc/INPUT_PW.html#idm735}{tstress} in \&control card \\ \hline
 
 %%%%%%%%%%%%% EDIFF %%%%%%%%%%%%%
 \href{https://www.vasp.at/wiki/index.php/EDIFF}{EDIFF} &
  % Notes
 Specifies the global break condition for the electronic SC-loop in units of eV. &
 % QE Replacement
 \href{https://www.quantum-espresso.org/Doc/INPUT_PW.html#idm771}{conv\_thr} in \&electrons card \\ \hline
 
 %%%%%%%%%%%%% EDIFFG %%%%%%%%%%%%%
 \href{https://www.vasp.at/wiki/index.php/EDIFFG}{EDIFFG} &
  % Notes
  Defines the break condition for the ionic relaxation loop. &
 % QE Replacement
 \href{https://www.quantum-espresso.org/Doc/INPUT_PW.html#idm117}{etot\_conv\_thr} in \&control card \\ \hline
 
 %%%%%%%%%%%%% LCHARG %%%%%%%%%%%%%
 \href{https://www.vasp.at/wiki/index.php/LCHARG}{LCHARG} &
  % Notes
 Determines whether the charge densities files (CHGCAR and CHG) are written &
 % QE Replacement
 Specify with \href{https://www.quantum-espresso.org/Doc/INPUT_PW.html#idm1487}{disk\_io} options in \&control card \\ \hline
 
  %%%%%%%%%%%%% LWAVE %%%%%%%%%%%%%
 \href{template.url}{LWAVE} &
  % Notes
 Determines whether the wavefunctions are written file at the end of a run &
 % QE Replacement
 Specify with \href{https://www.quantum-espresso.org/Doc/INPUT_PW.html#idm1487}{disk\_io} options in \&control card \\ \hline
\end{tabularx}
\end{table}
\end{center}

%%%%%%%%%%%%%%%%%%%%%%%%%%% INPUT FILE PARAMETERS 2 %%%%%%%%%%%%%%%%%%%%%%%%%%%%%%
\newpage
\begin{center}
\begin{table}[ht]
\begin{tabularx}{\linewidth}{|>{\RaggedRight}p{2.5cm}|x|x|}\hline
 VASP  & Notes & QE \\ \hline
 
  %%%%%%%%%%%%% POTIM %%%%%%%%%%%%%
 \href{https://www.vasp.at/wiki/index.php/POTIM}{POTIM} &
  % Notes
 Sets the time step in molecular dynamics or the step width in ionic relaxations. &
 % QE Replacement
 \href{https://www.quantum-espresso.org/Doc/INPUT_PW.html#idm36}{dt} in \&control card \\ \hline
  
 %%%%%%%%%%%%% ICHARG %%%%%%%%%%%%%
 \href{https://www.vasp.at/wiki/index.php/ICHARG}{ICHARG} &
  % Notes
 Determines how VASP constructs the initial charge density. &
 % QE Replacement
 \href{https://www.quantum-espresso.org/Doc/INPUT_PW.html#idm764}{startingwfc} in \&electrons card \\ \hline

 %%%%%%%%%%%%% LDAU %%%%%%%%%%%%%
 \href{https://www.vasp.at/wiki/index.php/LDAU}{LDAU} &
  % Notes
 Switches on DFT+U &
 % QE Replacement
 \href{https://www.quantum-espresso.org/Doc/INPUT_PW.html#idm1699}{\&HUBBARD} card. See \href{https://www.quantum-espresso.org/wp-content/uploads/2022/03/pw_user_guide.pdf}{DFT+U} section of user guide for specifics on compilation. \\ \hline
 
 %%%%%%%%%%%%% LDAUJ %%%%%%%%%%%%%
 \href{https://www.vasp.at/wiki/index.php/LDAUJ}{LDAUJ} &
  % Notes
 Specifies the strength of the effective on-site exchange interactions &
 % QE Replacement
 Specify in \href{https://www.quantum-espresso.org/Doc/INPUT_PW.html#idm1699}{\&HUBBARD} card  \\ \hline
 
 %%%%%%%%%%%%% LDAUL %%%%%%%%%%%%%
 \href{https://www.vasp.at/wiki/index.php/LDAUL}{LDAUL} &
  % Notes
 Specifies the $l$-quantum number for which the on-site interaction is added. &
 % QE Replacement
 See \href{https://www.quantum-espresso.org/wp-content/uploads/2022/03/pw_user_guide.pdf}{DFT+U} section of user guide for specifics on compilation. \\ \hline
 
 %%%%%%%%%%%%% LDAUU %%%%%%%%%%%%%
 \href{https://www.vasp.at/wiki/index.php/LDAUU}{LDAUU} &
  % Notes
 Specifies the strength of the effective on-site Coulomb interactions. &
 % QE Replacement
 Specify in \href{https://www.quantum-espresso.org/Doc/INPUT_PW.html#idm1699}{\&HUBBARD} card \\ \hline
 
 %%%%%%%%%%%%% LDAUPRINT %%%%%%%%%%%%%
 \href{https://www.vasp.at/wiki/index.php/LDAUPRINT}{LDAUPRINT} &
  % Notes
 Controls the verbosity of the DFT+U routines (i.e. Write occupancy matrix to the OUTCAR file). &
 % QE Replacement
 No direct replacement, occupancy matrix always written to output. \\ \hline
 
  %%%%%%%%%%%%% LDAUTYPE %%%%%%%%%%%%%
 \href{https://www.vasp.at/wiki/index.php/LDAUTYPE}{LDAUTYPE} &
  % Notes
 Specifies the DFT+U variant that will be used. &
 % QE Replacement
 Specify in \href{https://www.quantum-espresso.org/Doc/INPUT_PW.html#idm1699}{\&HUBBARD} card \\ \hline
 
 %%%%%%%%%%%%% LASPH %%%%%%%%%%%%%
 \href{https://www.vasp.at/wiki/index.php/LASPH}{LASPH} &
  % Notes
 Boolean for toggling non-spherical contributions related to the gradient of the density in the PAW spheres. &
 % QE Replacement
 No direct replacement. Must use gradient corrected pseudo-potential a priori \\ \hline
 
 %%%%%%%%%%%%% LMAXMIX %%%%%%%%%%%%%
 \href{https://www.vasp.at/wiki/index.php/LMAXMIX}{LMAXMIX} &
  % Notes
 Controls up to which $l$-quantum number the one-center PAW charge densities are passed through the charge density mixer and written to the CHGCAR file. &
 % QE Replacement
 Preform \href{https://www.quantum-espresso.org/Doc/INPUT_PROJWFC.html}{projwfc.x} post-processing calculation \\ \hline
 
 %%%%%%%%%%%%% NELM %%%%%%%%%%%%%
 \href{https://www.vasp.at/wiki/index.php/NELM}{NELM} &
  % Notes
 Sets the maximum number of electronic self-consistency steps. &
 % QE Replacement
 \href{https://www.quantum-espresso.org/Doc/INPUT_PW.html#idm78}{electron\_maxstep} in \&electrons card \\ \hline
 
 %%%%%%%%%%%%% LVHAR %%%%%%%%%%%%%
 \href{https://www.vasp.at/wiki/index.php/LVHAR}{LVHAR} &
  % Notes
 Toggles the XC term in the definition of the potential &
 % QE Replacement
 No direct replacement, but not needed in either code \\ \hline
 
 %%%%%%%%%%%%% LVTOT %%%%%%%%%%%%%
 \href{https://www.vasp.at/wiki/index.php/LVTOT}{LVTOT} &
  % Notes
 Determines whether the total local potential is written to file &
 % QE Replacement
 No direct replacement, but not needed in either code \\ \hline
 
\end{tabularx}
\end{table}
\end{center}

%%%%%%%%%%%%%%%%%%%%%%%%%%% INPUT FILE PARAMETERS 3 %%%%%%%%%%%%%%%%%%%%%%%%%%%%%%
\newpage
\begin{center}
\begin{table}[ht]
\begin{tabularx}{\linewidth}{|>{\RaggedRight}p{2.5cm}|x|x|}\hline
 VASP  & Notes & QE \\ \hline

%%%%%%%%%%%%% LORBMOM %%%%%%%%%%%%%
 \href{https://www.vasp.at/wiki/index.php/LORBMOM}{LORBMOM} &
  % Notes
 Use the projectors of the PAW potentials to calculate the orbital moment within the PAW spheres, and write them to file. &
 % QE Replacement
 \href{https://www.quantum-espresso.org/Doc/INPUT_PW.html#idm128}{lorbm} in \&control card \\ \hline
 
 %%%%%%%%%%%%% LAECHG %%%%%%%%%%%%%
 \href{https://www.vasp.at/wiki/index.php/LAECHG}{LAECHG} &
  % Notes
 the all-electron charge density will be reconstructed explicitly and written to files. &
 % QE Replacement
 No direct replacement, but not needed in either code \\ \hline
 
 %%%%%%%%%%%%% NELMIN %%%%%%%%%%%%%
 \href{https://www.vasp.at/wiki/index.php/NELMIN}{NELMIN} &
  % Notes
 specifies the minimum number of electronic self-consistency steps. &
 % QE Replacement
 No direct replacement \\ \hline
 
 %%%%%%%%%%%%% ISYM %%%%%%%%%%%%%
 \href{https://www.vasp.at/wiki/index.php/ISYM}{ISYM} &
  % Notes
 Determines how to treat symmetry. &
 % QE Replacement
 Can toggle symmetry using \href{https://www.quantum-espresso.org/Doc/INPUT_PW.html#idm768}{nosym} in the \&system card \\ \hline
 
 %%%%%%%%%%%%% ALGO %%%%%%%%%%%%%
 \href{https://www.vasp.at/wiki/index.php/ALGO}{ALGO} &
  % Notes
 Specify the electronic minimization algorithm &
 % QE Replacement
 \href{https://www.quantum-espresso.org/Doc/INPUT_PW.html#idm768}{diagonalization} in \&electrons card \\ \hline

\end{tabularx}
\end{table}
\end{center}
\end{document}

%%%%%%%%%%%%%% Template %%%%%%%%%%%%%
%  \href{templateurl}{Template} &
%   % Notes
%  notes &
%  % QE Replacement
%  TODO \\ \hline
